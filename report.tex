\documentclass[a4paper, twocolumn, 10pt, revision]{ess}
\usepackage{dmsc}
\usepackage{paralist}
%\usepackage{esslogo}
\input{generated/revision}

\addbibresource{STAP-Reports.bib}

\newcommand{\pdfsubject}{Spectroscopy Scientific and Technical Advisory Panel}
\newcommand{\pdftitle}{Instrument Data Scientist Report}
\newcommand{\pdfauthor}{Gregory Tucker}
\newcommand{\pdfaffil}{European Spallation Source DMSC}
\hypersetup{%
  pdfinfo={%
    Title=\pdftitle,%
    Subject=\pdfsubject,%
    Author=\pdfauthor,%
    Revision=\revision,%
    Keywords={Revision:\revision,Affiliation:\pdfaffil}%
  }%
}

% set extra footer information
% \ifoot{\pdfauthor\\\pdfaffil}
% \ofoot{\pdftitle\\\revision}
% \cfoot{}


% \pagestyle{scrheadings}


\begin{document}
\titlehead{\pdfsubject \hfill Revision: \revision}
%\subject{\pdfsubject}
\title{\pdftitle}
\author{\pdfauthor}
\date{\revisiondate}
\maketitle

\begin{abstract}
  Abstract
\end{abstract}

\section{Simulations}

\section{User tools}
It is undoubtable that some tasks can be made easier through graphical user interfaces (GUI) to scripts and other routines.
Often a user may desire a tool to help carry out a measurement, which must be flexible enough to support their specific use case.
Traditional GUI deveopment can be time-consuming, is typically platform dependent, and is almost-always highly-inflexible.
Therefore traditional GUIs would be a bad choice to provide an interface to one-off or quickly-developed tools.

Development in Python
Use of Pyodide


\section{Project planning}



\section{Pixel mapping}
Interest was expressed about pixel mapping in the last ESS Spectroscopy STAP Report \autocite{spectroscopy_stap_2022}.
Extensive per-instrument details are available from the Experiment Control and Data Curation (ECDC)
\href{https://confluence.esss.lu.se/display/ECDC/Instrument+Status+Overview}{Instrument Status Overview} page within 
ESS Confluence.
The CSPEC documentation available there as of this moment does not yet reflect the change to $^3$He detectors,
but should instead be similar to the \href{https://project.esss.dk/owncloud/index.php/s/4M60TNdqkMcppUX}{BIFROST documentation}.

A high-level summary of the planned readout scheme follows:
\begin{inparaenum}
\item high-voltage signals from the ends of a wire, $A$ and $B$, are digitized by one of series of Front End Nodes;
\item a Readout Master collects the digitized signals, bundles them into network packets along with timing information, and sends the packets over a redundant  connection to a server room located in the Central Utility Building, H01, on the ESS site;
\item an instance of the \href{https://github.com/ess-dmsc/event-formation-unit}{Event Formation Unit} (EFU) software uses linear charge division to pixelate each wire.
\end{inparaenum}
The identification of pixels from the continuous charge-division signal, $x = A/(A+B)$, will require at least two values per tube to identify its ends -- in the case of BIFROST this is six values per wire due to the triple in-series tubes.
The EFU extracts $x$ between the specified end-points and optionally applies a nonlinear correction with calibrated polynomial coefficients
before subdividing it into a configurable number of pixels.
The number of EFU pixels should match the position resolution of the tube, and 100 pixels per tube are expected for BIFROST.

Since the ESS data transformation for spectroscopy will keep neutron event data and avoid unnecessarily producing histograms, 
combining pixels to, e.g., match instrumental resolution conditions, will only be performed preceding the final histogram-creating step.



\printbibliography
\end{document}
